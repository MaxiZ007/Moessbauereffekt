

\documentclass[a4paper, 10pt]{scrreprt}

%%%%%%%% Fonts and Encoding
\usepackage[ngerman]{babel}
\usepackage[utf8]{inputenc}
\usepackage[sc]{mathpazo}		%Palatino Font
\usepackage[svgnames]{xcolor}	%svgnames for RGB colors
\usepackage[T1]{fontenc}
\linespread{1.44}				%only with Palatino
\usepackage{microtype}
\DisableLigatures{encoding=T1,family=tt*}

%%%%%%%% Standard Packages
\usepackage{fancyhdr, graphicx, amsmath, amssymb, amsthm, bm, mathrsfs, paralist, shadethm, xspace}
%%%%%%%%
%\graphicspath{{./Bilder/}{./Bilder/Raman/}{./Bilder/Technik/}{./Bilder/Messungen/}{./Bilder/Auswertung/}{./Bilder/Titel/}{./Bilder/BaFeAs/}}
%%%%%%%%
\usepackage{subfig} % make it possible to include more than one captioned figure/table in a single float
\usepackage{multirow}
%%%%%%%% Hyperref Setup
\usepackage[pdftex, colorlinks=true,linkcolor=DarkBlue, urlcolor=black, citecolor=DarkGreen]{hyperref}
%%%%%%%
%\renewcommand{\thefigure}{\thesection.\arabic{figure}} 
%\renewcommand{\thetable}{\thesection.\arabic{table}} 

%%%%%%%% Andi...
\usepackage[inline]{enumitem}
\usepackage{geometry}
\geometry{bottom=1.5in}

\usepackage{nicefrac}

\usepackage{tikz}
\usetikzlibrary{arrows}
\usetikzlibrary{decorations.pathmorphing}
\usetikzlibrary{intersections}
\usetikzlibrary{calc}
\usetikzlibrary{patterns}
\usetikzlibrary{backgrounds}

\hypersetup{pdftitle={Titel}, pdfauthor={Ziegler, Maximilian}, pdfsubject={irgendwas}, pdfkeywords={Technische Universität München}, pageanchor=true}

%%%%%%%% Header and Footer (fancy)
\pagestyle{fancy}
\fancyhf{}
\renewcommand{\headrule}{{\color{gray} \hrule width\headwidth height\headrulewidth \vskip-\headrulewidth}}
\setlength{\headheight}{15pt}
%\fancyhead[R]{\color{gray}\slshape \nouppercase\rightmark}
\fancyhead[L]{\color{gray}\slshape \nouppercase\leftmark}
\fancyfoot[C]{\color{gray}\oldstylenums{\thepage}}

\fancypagestyle{plain} 
\fancyhf{}
\fancyfoot[C]{\color{gray}\oldstylenums{\thepage}}\renewcommand{\headrulewidth}{0pt} 

%%%%%%%% Newcommand 
\newcommand{\bfa}{BaFe\textsubscript{2}As\textsubscript{2}\xspace}
\newcommand{\bfca}{Ba(Fe\textsubscript{1-x}Co\textsubscript{x})\textsubscript{2}As\textsubscript{2}\xspace}
%%%%%%%%
%%%%%%%% Document Information
\author{Maximilian Ziegler}
\title{Titel}
\date{\today}

%%%%%%%% Main Document
\begin{document}

%%%%%%%% Title Page
\begin{titlepage}
%\setcounter{page}{0}
\clearpage \thispagestyle{empty} \setcounter{page}{0}
%\renewcommand{\arraystretch}{1}%     Felder der Tabellen wieder verkleinern

\begin{center}
\begin{figure}[h]
\centering

\end{figure}
\vspace{1.5cm}

%



%\end{center}
%
%\begin{figure}[H]
%\centering
{\centering
	{\huge 


		Fortgeschrittenenpraktikum							\\
		\vspace{2cm}
		
		{\bf	Mößbauereffekt	}										\\
		\vspace{1.5cm}
		
		Ausarbeitung												\\
		\vspace{1cm}
		
		Robin Häcker, Matrikel 03626146						\\
		\vspace{1cm}
		
		Philipp Klose, Matrikel 03631983						\\
		\vspace{1cm}
		
		Maximilian Ziegler, Matrikel 03638495						\\
		\vspace{1cm}
		
		22.10.2013												\\
		\vspace{1cm}
		
		Gruppe 136													\\
	}}
%\end{figure}
\end{center}
\end{titlepage}
%
\clearpage \thispagestyle{empty} \setcounter{page}{0}
\newpage \thispagestyle{empty}
\tableofcontents
\newpage
%%%%%%%%
\pagenumbering{arabic}

\chapter{Einleitung}
\label{cha:Einleitung}



\end{document}